\documentclass{num_exercise}

% required for localization (ngerman prefered for german loc.)
\usepackage[english]{babel}

% only required for this example
\usepackage{blindtext}

% some metadata, most of it is optional
\lecture{Example Lecture Title}
\lecturer{Prof. Dr. Example Lecturer $\cdot$  M. Sc. Teaching Assistant}
\term{Summer semester 2017}
\sheettitle{Exercise Sheet 1.}
\sheetdate{\today}
\university{Institute of Mathematics, Example University, 12345 Example Town}
\website{https://github.com/nobbs/num-exercise}
\lastline{Some last information can go here, e.g. where to drop off the exercises.}

\begin{document}
  \blindtext

  % see the exsheets documentation for more information on how to use these
  % environment and the other features
  \begin{question}[subtitle=Some not so easy tasks]{10+2}
  \begin{enumerate}
    \item (\points{2}) \blindtext
    \item (\points{2}) \blindtext
    \begin{enumerate}
      \item (\points{3}) \blindtext
      \item (+\addpoints{1+7}) \blindtext
    \end{enumerate}
  \end{enumerate}
    \blindtext[1]
  \end{question}

  \begin{solution}[print]
    \blindtext[2]
  \end{solution}  

  \begin{question}{3}
    \blindtext[1]
  \end{question}  

  \begin{question}[subtitle=A bonus task]{+2}
    \blindtext
  \end{question}
\end{document}